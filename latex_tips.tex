% Created 2015-09-22 Tue 14:27
\documentclass[12pt]{extarticle}
\usepackage{a4}
\usepackage{color}
\definecolor{hotpink}{rgb}{0.9,0,0.5}
\usepackage[colorlinks,urlcolor=blue,citecolor=hotpink,linkcolor=blue]{hyperref}
\author{Nick Higham\footnote{School of Mathematics, University of Manchester, Manchester, M13 9PL, UK (nick.higham@manchester.ac.uk, http://www.maths.manchester.ac.uk/\string~higham}}
\date{\today}
\title{Top Tips for New \LaTeX{} Users}
\hypersetup{
  pdfkeywords={},
  pdfsubject={},
  pdfcreator={Emacs 25.0.50.1 (Org mode 8.2.10)}}
\begin{document}

\maketitle

\label{sec-1}

This article is aimed at relatively new \LaTeX\ users.
It is written particularly for my own students, with the aim of helping them
to avoid making common errors.

This article exists in two forms: a WordPress blog post and a PDF file
generated by \LaTeX, both produced from the same Emacs Org file.
Since Wordpress does not handle \LaTeX\ very well I recommend reading the
PDF version.

\subsection*{1. New Paragraphs}
\label{sec-1-1}
In \LaTeX\ a new paragraph is started  by leaving a blank line.

Do not start a new paragraph by using \texttt{\textbackslash{}\textbackslash{}}
(it merely terminates a line).
Indeed you should almost never type \texttt{\textbackslash{}\textbackslash{}}, except within 
environments such as \texttt{array}, \texttt{tabular}, and so on.

\subsection*{2. Math Mode}
\label{sec-1-2}
Always type mathematics in math mode (as \texttt{\$..\$} or \texttt{\textbackslash{}(..\textbackslash{})}), 
to produce ``$y = f(x)$" instead of ``y = f(x)",
and ``the dimension $n$"
instead of ``the dimension n".
For displayed equations use \texttt{\$\$}, \texttt{\textbackslash{}[..\textbackslash{}]}, or one of the display
environments (see Section 7).

Punctuation should appear outside math mode, for inline equations,
otherwise the spacing will be
incorrect. Here is an example.


\begin{itemize}
\item Correct: \texttt{The variables \$x\$, \$y\$, and \$z\$ satisfy \$x\textasciicircum{}2 + y\textasciicircum{}2 = z\textasciicircum{}2\$.}
\item Incorrect: \texttt{The variables \$x,\$ \$y,\$ and \$z\$ satisfy \$x\textasciicircum{}2 + y\textasciicircum{}2 = z\textasciicircum{}2.\$}
\end{itemize}

For displayed equations, punctuation should appear as part of the display.
All equations \emph{must} be punctuated, as they are part of a sentence.

\subsection*{3. Mathematical Functions in Roman}
\label{sec-1-3}
Mathematical functions should be typeset in roman font.
This is done automatically for the many standard mathematical functions that 
\LaTeX\ supports, such as 
\texttt{\textbackslash{}sin},
\texttt{\textbackslash{}tan},
\texttt{\textbackslash{}exp},
\texttt{\textbackslash{}max},
etc.

If the function you need is not built into \LaTeX, create your own.
The easiest way to do this is to use the \texttt{amsmath} package and type,
for example,
\begin{verbatim}
\usepackage{amsmath}
...
% In the preamble.
\DeclareMathOperator{\diag}{diag}  
\DeclareMathOperator{\inert}{Inertia}
\end{verbatim}
Alternatively, if you are not using the \texttt{amsmath} package you can type
\begin{verbatim}
\def\diag{\mathop{\mathrm{diag}}}
\end{verbatim}

\subsection*{4. Maths Expressions}
\label{sec-1-4}
Ellipses (dots) are never explicitly typed as ``\ldots{}".
Instead they are typed as \texttt{\textbackslash{}dots} for baseline dots,
as in 
\texttt{\$x\_1,x\_2,\textbackslash{}dots,x\_n\$}
(giving $x_1,x_2,\dots,x_n$)
or as \texttt{\textbackslash{}cdots} for vertically centered dots, as in 
\texttt{\$x\_1 + x\_2 + \textbackslash{}cdots + x\_n\$}
(giving $x_1 + x_2 + \cdots + x_n$).

Type \texttt{\$i\$th} 
(giving $i$th)
instead of \texttt{\$i'th\$} (giving $i\rq th$)
or \texttt{\$i\textasciicircum{}\{th\}\$} (giving $i^{th}$).

(For some subtle aspects of the use of ellipses,
see \href{https://nickhigham.wordpress.com/2014/02/06/how-to-typeset-an-ellipsis-in-a-mathematical-expression}{How To Typeset an Ellipsis in a Mathematical Expression}.)

Avoid using \texttt{\textbackslash{}frac} to produce stacked fractions in the text.  Write
$n^3/3$ flops instead of $\frac{n^3}{3}$ flops.

For ``much less than", type \verb"$\ll$", giving $\ll$, 
not \verb"<<", which gives $<<$.
Similarly, "much greater than" is typed as \texttt{\textbackslash{}gg}, giving $\gg$.
If you are using angle brackets to denote an inner product
use \texttt{\textbackslash{}langle} and \texttt{\textbackslash{}rangle}:

\begin{itemize}
\item incorrect: $<x,y>$, typed as  \texttt{\$<x,y>\$}.
\item correct: $\langle x,y \rangle$, typed as \texttt{\$\textbackslash{}langle x,y \textbackslash{}rangle\$}
\end{itemize}

\subsection*{5. Text in Displayed Equations}
\label{sec-1-5}
When a displayed equation contains text such as 
``subject to $x \ge 0$'',
instead of putting the text in \texttt{\textbackslash{}mathrm} put the text in an
\texttt{\textbackslash{}mbox}, as in 
\texttt{\textbackslash{}mbox\{subject to \$x \textbackslash{}ge 0\$\}}.
Note that \texttt{\textbackslash{}mbox} switches out of math mode,
and this has the advantage of ensuring the correct spacing between words.
If you are using the amsmath package you can use the \texttt{\textbackslash{}text} command
instead of \texttt{\textbackslash{}mbox}.
\subsubsection*{Example}
\label{sec-1-5-1}
\begin{verbatim}
$$
      \min\{\, \|A-X\|_F: \mbox{$X$ is a correlation matrix} \,\}.
$$
\end{verbatim}
produces
$$
   \min\{\, \|A-X\|_F: \mbox{$X$ is a correlation matrix} \,\}.
$$
\subsection*{6. BibTeX}
\label{sec-1-6}
Produce your bibliographies using BibTeX, creating your own bib file.
Note three important points.

\begin{itemize}
\item ``Export citation" options on journal websites rarely produce perfect bib
entries.  More often than not the entry has an improperly cased title,
an incomplete or incorrectly accented author name, improperly typeset
maths in the title, or some other error, so always check and improve
the entry.

\item If you wish to cite one of my papers download the latest version of
\href{https://github.com/higham/njhigham-bib}{\texttt{njhigham.bib}} (along with \texttt{strings.bib} supplied with it) and include it in
your \texttt{\textbackslash{}bibliography} command.

\item Decide on a consistent format for your bib entry keys and stick to it. 
In the format used in the Numerical Linear Algebra group at Manchester a
2010 paper by Smith and Jones has key \texttt{smjo10}, a 1974 book by Aho,
Hopcroft, and Ullman has key \texttt{ahu74}, while a 1990 book by Smith has key
\texttt{smit90}.
\end{itemize}

\subsection*{7. Spelling Errors and \LaTeX\ Errors}
\label{sec-1-7}
There is no excuse for your writing to contain spelling errors, given the
wide availability of spell checkers.  You'll need a spell checker that
understands \LaTeX\ syntax.

There are also tools for checking \LaTeX\ syntax.
One that comes with \href{https://www.tug.org/texlive/}{\TeX{} Live} is \href{https://www.ctan.org/tex-archive/support/lacheck?lang=en}{\texttt{lacheck}}, which describes itself as 
``a consistency checker for \LaTeX{} documents''.
Such a tool can point out possible syntax errors, or semantic errors
such as unmatched parentheses, and warn of common mistakes.

\subsection*{8. Quotation Marks}
\label{sec-1-8}
\LaTeX\ has a left quotation mark, \verb"`", and 
a right quotation mark, \verb"'", typed as the single left and right quotes on the
keyboard, respectively.
A left or right double quotation mark is produced by typing two single
quotes of the appropriate type.
The double quotation mark itself produces the same as two right quotation marks.
Example: ``hello'' is typed as \verb!``hello''! and not as
\verb!"hello"!, which produces ''hello''.

\subsection*{9. Captions}
\label{sec-1-9}
Captions go \emph{above} tables but \emph{below} figures.
So put the \texttt{caption} command at the start of a \texttt{table} environment but at
the end of a \texttt{figure} environment.
The \texttt{\textbackslash{}label} statement should go after the \texttt{\textbackslash{}caption} statement
(or it can be put inside it), otherwise references to that label will refer
to the subsection in which the label appears rather than the figure or table.

\subsection*{10. Tables}
\label{sec-1-10}
\LaTeX\ makes it easy to put many rules, some of them double, 
in and around a table,
using 
\texttt{\textbackslash{}cline}, \texttt{\textbackslash{}hline}, and the \texttt{|} column formatting symbol.
However, it is good style to minimize the number of rules.
A common task for journal copy editors is to remove rules from tables in 
submitted manuscripts.

\subsection*{11. Source Code}
\label{sec-1-11}
\LaTeX\ source code should be laid out so that it is readable, in order to
aid editing and debugging, to help you to understand the code when you return
to it after a break, and to aid collaborative writing.
Readability means that logical structure should be apparent,
in the same way as when indentation is used in writing a computer program.
In particular, it is is a good idea 
to start new sentences on new lines, which makes it easier to cut
and paste them during editing,
and also makes a diff of two versions of the file more readable.

\subsubsection*{Example:}
\label{sec-1-11-1}

Good:
\begin{verbatim}
$$
U(\zbar) = U(-z) = 
            \begin{cases}
                -U(z),   & z\in D, \\ 
                -U(z)-1, & \mbox{otherwise}.
            \end{cases}
$$
\end{verbatim}
Bad:
\begin{verbatim}
$$U(\zbar) = U(-z) = 
\begin{cases}-U(z),   & z\in D, \\ 
-U(z)-1, & \mbox{otherwise}.
\end{cases}$$
\end{verbatim}

\subsection*{12. Multiline Displayed Equations}
\label{sec-1-12}
For displayed equations occupying more than one line it is best to use the
environments provided by the amsmath package.
Of these, \texttt{align} (and \texttt{align*} if equation numbers are not wanted)
is the one I use almost all the time.
Example:
\begin{verbatim}
\begin{align*}
  \cos(A) &= I - \frac{A^2}{2!} + \frac{A^4}{4!} + \cdots,\\
  \sin(A) &= A - \frac{A^3}{3!} + \frac{A^5}{5!} -  \cdots,
\end{align*}
\end{verbatim}
Others, such as \texttt{gather} and \texttt{aligned}, are occasionally needed.

Avoid using the standard \LaTeX\ environment 
\texttt{eqnarray}, because it 
doesn't produce as good results as the amsmath environments,
nor is it as versatile.
For more details see the article \href{http://tug.org/pracjourn/2006-4/madsen}{Avoid Eqnarray}.

\subsection*{13. Synonyms}
\label{sec-1-13}
This final category concerns synonyms and is a matter of personal
preference.
I prefer \texttt{\textbackslash{}ge} and \texttt{\textbackslash{}le} to the equivalent \texttt{\textbackslash{}geq} \texttt{\textbackslash{}leq\textbackslash{}}
(why type the extra characters?).


I also prefer to use \texttt{\$..\$} for math mode instead of \texttt{\textbackslash{}(..\textbackslash{})} and 
                     \texttt{\$\$..\$\$} for display math mode instead of \texttt{\textbackslash{}[..\textbackslash{}]}.
My preferences are the original \TeX\ syntax, while the alternatives were
introduced by \LaTeX.
The slashed forms are obviously easier to parse, but this is
one case where I prefer to stick with tradition.
If dollar signs are good enough for Don Knuth, they are good enough for me!

I don't think anybody uses \LaTeX's verbose
\begin{verbatim}
\begin{math}..\end{math}
\end{verbatim}
or
\begin{verbatim}
\begin{displaymath}..\end{displaymath}
\end{verbatim}
Also note that 
\texttt{\textbackslash{}begin\{equation*\}..\textbackslash{}end\{equation*\}} 
(for unnumbered equations) exists in the amsmath package
but not in in \LaTeX\ itself.
% Emacs 25.0.50.1 (Org mode 8.2.10)
\end{document}
